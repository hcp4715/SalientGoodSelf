% Options for packages loaded elsewhere
\PassOptionsToPackage{unicode}{hyperref}
\PassOptionsToPackage{hyphens}{url}
%
\documentclass[
  man]{apa6}
\usepackage{amsmath,amssymb}
\usepackage{lmodern}
\usepackage{iftex}
\ifPDFTeX
  \usepackage[T1]{fontenc}
  \usepackage[utf8]{inputenc}
  \usepackage{textcomp} % provide euro and other symbols
\else % if luatex or xetex
  \usepackage{unicode-math}
  \defaultfontfeatures{Scale=MatchLowercase}
  \defaultfontfeatures[\rmfamily]{Ligatures=TeX,Scale=1}
\fi
% Use upquote if available, for straight quotes in verbatim environments
\IfFileExists{upquote.sty}{\usepackage{upquote}}{}
\IfFileExists{microtype.sty}{% use microtype if available
  \usepackage[]{microtype}
  \UseMicrotypeSet[protrusion]{basicmath} % disable protrusion for tt fonts
}{}
\makeatletter
\@ifundefined{KOMAClassName}{% if non-KOMA class
  \IfFileExists{parskip.sty}{%
    \usepackage{parskip}
  }{% else
    \setlength{\parindent}{0pt}
    \setlength{\parskip}{6pt plus 2pt minus 1pt}}
}{% if KOMA class
  \KOMAoptions{parskip=half}}
\makeatother
\usepackage{xcolor}
\usepackage{graphicx}
\makeatletter
\def\maxwidth{\ifdim\Gin@nat@width>\linewidth\linewidth\else\Gin@nat@width\fi}
\def\maxheight{\ifdim\Gin@nat@height>\textheight\textheight\else\Gin@nat@height\fi}
\makeatother
% Scale images if necessary, so that they will not overflow the page
% margins by default, and it is still possible to overwrite the defaults
% using explicit options in \includegraphics[width, height, ...]{}
\setkeys{Gin}{width=\maxwidth,height=\maxheight,keepaspectratio}
% Set default figure placement to htbp
\makeatletter
\def\fps@figure{htbp}
\makeatother
\setlength{\emergencystretch}{3em} % prevent overfull lines
\providecommand{\tightlist}{%
  \setlength{\itemsep}{0pt}\setlength{\parskip}{0pt}}
\setcounter{secnumdepth}{-\maxdimen} % remove section numbering
% Make \paragraph and \subparagraph free-standing
\ifx\paragraph\undefined\else
  \let\oldparagraph\paragraph
  \renewcommand{\paragraph}[1]{\oldparagraph{#1}\mbox{}}
\fi
\ifx\subparagraph\undefined\else
  \let\oldsubparagraph\subparagraph
  \renewcommand{\subparagraph}[1]{\oldsubparagraph{#1}\mbox{}}
\fi
\newlength{\cslhangindent}
\setlength{\cslhangindent}{1.5em}
\newlength{\csllabelwidth}
\setlength{\csllabelwidth}{3em}
\newlength{\cslentryspacingunit} % times entry-spacing
\setlength{\cslentryspacingunit}{\parskip}
\newenvironment{CSLReferences}[2] % #1 hanging-ident, #2 entry spacing
 {% don't indent paragraphs
  \setlength{\parindent}{0pt}
  % turn on hanging indent if param 1 is 1
  \ifodd #1
  \let\oldpar\par
  \def\par{\hangindent=\cslhangindent\oldpar}
  \fi
  % set entry spacing
  \setlength{\parskip}{#2\cslentryspacingunit}
 }%
 {}
\usepackage{calc}
\newcommand{\CSLBlock}[1]{#1\hfill\break}
\newcommand{\CSLLeftMargin}[1]{\parbox[t]{\csllabelwidth}{#1}}
\newcommand{\CSLRightInline}[1]{\parbox[t]{\linewidth - \csllabelwidth}{#1}\break}
\newcommand{\CSLIndent}[1]{\hspace{\cslhangindent}#1}
\ifLuaTeX
\usepackage[bidi=basic]{babel}
\else
\usepackage[bidi=default]{babel}
\fi
\babelprovide[main,import]{english}
% get rid of language-specific shorthands (see #6817):
\let\LanguageShortHands\languageshorthands
\def\languageshorthands#1{}
% Manuscript styling
\usepackage{upgreek}
\captionsetup{font=singlespacing,justification=justified}

% Table formatting
\usepackage{longtable}
\usepackage{lscape}
% \usepackage[counterclockwise]{rotating}   % Landscape page setup for large tables
\usepackage{multirow}		% Table styling
\usepackage{tabularx}		% Control Column width
\usepackage[flushleft]{threeparttable}	% Allows for three part tables with a specified notes section
\usepackage{threeparttablex}            % Lets threeparttable work with longtable

% Create new environments so endfloat can handle them
% \newenvironment{ltable}
%   {\begin{landscape}\centering\begin{threeparttable}}
%   {\end{threeparttable}\end{landscape}}
\newenvironment{lltable}{\begin{landscape}\centering\begin{ThreePartTable}}{\end{ThreePartTable}\end{landscape}}

% Enables adjusting longtable caption width to table width
% Solution found at http://golatex.de/longtable-mit-caption-so-breit-wie-die-tabelle-t15767.html
\makeatletter
\newcommand\LastLTentrywidth{1em}
\newlength\longtablewidth
\setlength{\longtablewidth}{1in}
\newcommand{\getlongtablewidth}{\begingroup \ifcsname LT@\roman{LT@tables}\endcsname \global\longtablewidth=0pt \renewcommand{\LT@entry}[2]{\global\advance\longtablewidth by ##2\relax\gdef\LastLTentrywidth{##2}}\@nameuse{LT@\roman{LT@tables}} \fi \endgroup}

% \setlength{\parindent}{0.5in}
% \setlength{\parskip}{0pt plus 0pt minus 0pt}

% Overwrite redefinition of paragraph and subparagraph by the default LaTeX template
% See https://github.com/crsh/papaja/issues/292
\makeatletter
\renewcommand{\paragraph}{\@startsection{paragraph}{4}{\parindent}%
  {0\baselineskip \@plus 0.2ex \@minus 0.2ex}%
  {-1em}%
  {\normalfont\normalsize\bfseries\itshape\typesectitle}}

\renewcommand{\subparagraph}[1]{\@startsection{subparagraph}{5}{1em}%
  {0\baselineskip \@plus 0.2ex \@minus 0.2ex}%
  {-\z@\relax}%
  {\normalfont\normalsize\itshape\hspace{\parindent}{#1}\textit{\addperi}}{\relax}}
\makeatother

% \usepackage{etoolbox}
\makeatletter
\patchcmd{\HyOrg@maketitle}
  {\section{\normalfont\normalsize\abstractname}}
  {\section*{\normalfont\normalsize\abstractname}}
  {}{\typeout{Failed to patch abstract.}}
\patchcmd{\HyOrg@maketitle}
  {\section{\protect\normalfont{\@title}}}
  {\section*{\protect\normalfont{\@title}}}
  {}{\typeout{Failed to patch title.}}
\makeatother

\usepackage{xpatch}
\makeatletter
\xapptocmd\appendix
  {\xapptocmd\section
    {\addcontentsline{toc}{section}{\appendixname\ifoneappendix\else~\theappendix\fi\\: #1}}
    {}{\InnerPatchFailed}%
  }
{}{\PatchFailed}
\keywords{Perceptual decision-making, Self positivity bias, moral character\newline\indent Word count: X}
\DeclareDelayedFloatFlavor{ThreePartTable}{table}
\DeclareDelayedFloatFlavor{lltable}{table}
\DeclareDelayedFloatFlavor*{longtable}{table}
\makeatletter
\renewcommand{\efloat@iwrite}[1]{\immediate\expandafter\protected@write\csname efloat@post#1\endcsname{}}
\makeatother
\usepackage{lineno}

\linenumbers
\usepackage{csquotes}
\usepackage{rotating}
\DeclareDelayedFloatFlavor{sidewaysfigure}{figure}
\ifLuaTeX
  \usepackage{selnolig}  % disable illegal ligatures
\fi
\IfFileExists{bookmark.sty}{\usepackage{bookmark}}{\usepackage{hyperref}}
\IfFileExists{xurl.sty}{\usepackage{xurl}}{} % add URL line breaks if available
\urlstyle{same} % disable monospaced font for URLs
\hypersetup{
  pdftitle={The good person is me: Spontaneous self-referential process prioritizes moral character in perceptual matching},
  pdfauthor={Hu Chuan-Peng1, 2, Kaiping Peng2, \& Jie Sui3},
  pdflang={en-EN},
  pdfkeywords={Perceptual decision-making, Self positivity bias, moral character},
  hidelinks,
  pdfcreator={LaTeX via pandoc}}

\title{The good person is me: Spontaneous self-referential process prioritizes moral character in perceptual matching}
\author{Hu Chuan-Peng\textsuperscript{1, 2}, Kaiping Peng\textsuperscript{2}, \& Jie Sui\textsuperscript{3}}
\date{}


\shorttitle{Priorization of good character is self-referential}

\authornote{

Hu Chuan-Peng, School of Psychology, Nanjing Normal University, 210024 Nanjing, China.
Kaiping Peng, Department of Psychology, Tsinghua University, 100084 Beijing, China.
Jie Sui, School of Psychology, University of Aberdeen, Aberdeen, Scotland.
Authors contriubtion: HCP, JS, \& KP design the study, HCP collected the data, HCP analyzed the data and drafted the manuscript. All authors read and agreed upon the current version of the manuscripts.

Correspondence concerning this article should be addressed to Hu Chuan-Peng, School of Psychology, Nanjing Normal University, Ninghai Road 122, Gulou District, 210024 Nanjing, China. E-mail: \href{mailto:hcp4715@hotmail.com}{\nolinkurl{hcp4715@hotmail.com}}

}

\affiliation{\vspace{0.5cm}\textsuperscript{1} Nanjing Normal University, 210024 Nanjing, China\\\textsuperscript{2} Tsinghua University, 100084 Beijing, China\\\textsuperscript{3} University of Aberdeen, Aberdeen, Scotland}

\abstract{%
Moral character is central to social evaluation and moral judgment. However, whether moral character information is prioritized in perceptual decision-making was debated. Here we investigated the effect of moral character on perceptual decision-making through an associative learning task. Participants first learned associations between different geometric shapes and moral characters and then performed a simple perceptual matching task. Across five experiments (N = 192 ), we found a robust prioritization effect of good character-related information, i.e., participants responded faster and more accurately to geometric shapes that were associated with good characters than shapes associated with neutral or bad characters. We then examine whether the prioritization of good character was due to valence or self-reference. Data from three experiments (N = 108) demonstrated that the prioritization effect of good character was robust only when the good character referred to the self but weak or non-exist when it referred to others. Additional two experiments (N = 104) further revealed that the mutual facilitation between good character and self-reference occurred even when one of them was task-irrelevant. Together, these results suggested a spontaneous self-referential process as a mechanism of the prioritization effect of good character.
}



\begin{document}
\maketitle

Alternative title: Self-relevance modulates the prioritization of the good character in perceptual matching

\hypertarget{introduction}{%
\section{Introduction}\label{introduction}}

{[}quotes about moral character{]}

Is moral information prioritized in perception? This question evoked much heat a few years ago but remains unsolved. On the one hand, morality is a basic dimension in social evaluation (Dunbar, 2004; Ellemers, 2018; Goodwin, 2015; Goodwin, Piazza, \& Rozin, 2014), this importance should grant moral information more salient than morally neutral information and thus prioritized when the attentional resource is limited. This logic is similar to other stimuli that are also important to humans, e.g., threatening stimuli {[}XX{]}, rewards {[}XX{]}, or self-related stimuli {[}XX{]}. Indeed, previous studies reported bad characters are prioritized in visual processing (Anderson, Siegel, Bliss-Moreau, \& Barrett, 2011; Eiserbeck \& Abdel Rahman, 2020), suggesting that bad people are detected faster than neutral or good people. On the other hand, there is evidence against the view that morally bad information is prioritized in perception. First, researchers reported positive bias in processing moral-related information. For example, Shore and Heerey (2013) found that faces with positive interaction in a trust game were prioritized in the pre-attentive process. Second, the negative bias in perceiving moral information is not robust (Stein, Grubb, Bertrand, Suh, \& Verosky, 2017). Third, the mechanism underlying the reported negative bias in processing moral-related information is debated {[}XX{]}. In short, while the importance of morality is widely recognized, whether moral information is prioritized in perceptual decision-making is still an open question. Here we manipulated the moral character by an associated learning task and investigated whether immediately acquired moral character information is prioritized in a perceptual matching task.

If moral character information is indeed prioritized, the next question is how? Previous studies explain the effect based on valence. For example, the negative bias toward moral information is explained by aligning moral information with affective stimuli and threat detection was supposed to be the potential mechanism {[}XXX{]}. The positive bias toward moral information, on the other hand, is explained by value-based attention {[}XXX{]}. However, these explanations often ignore the fact the value is subjective \emph{per se} (Juechems \& Summerfield, 2019). Merely associating with the self can prioritize the stimuli in perception, attention, working memory, and long-term memory Sui \& Humphreys (2015). Here, we explicitly included self-relevance in our experimental design and tested whether the prioritization of moral character is modulated by self-relevance. We adopted an associative learning task, or self-tagging task, which has been widely used in studying the self-relevance effect. It is based on the well-established fact that humans can quickly learn the associations between symbols via language and change subsequent behaviors accordingly. This associative learning was not only used in studying the self-relevance effect but also other factors such as aversive stimuli {[}XX{]} and rewards {[}XX{]}. By explicitly instructing participants on which moral character is self-referencing and which is not, we can test whether the prioritization of moral character is by valence per se or by the self-referential of moral valence.

We address these questions by investigating how immediately acquired moral character information modulates the processing of neutral geometric shapes in a perceptual matching task. Unlike previous studies relies on faces or words as materials, stimuli used in the social associative task are geometric shapes, which acquire moral meaning before the perceptual matching task. Moreover, associations between shapes and different labels of moral characters are counter-balanced between participants, thus eliminating confounding effects by stimuli. Also, because we only used a few stimuli and they were repeatedly presented during the task, the results can not be explained by semantic priming (Unkelbach, Alves, \& Koch, 2020), which is the center of the debate on previous results (Firestone \& Scholl, 2015, 2016; Gantman \& Bavel, 2015, 2016; Jussim, Crawford, Anglin, Stevens, \& Duarte, 2016). We examined whether participants' performance in the perceptual matching task was altered by the immediately acquired moral character of the shapes --- in particular, whether the shapes associated with good or bad character are prioritized. We found a robust effect that shapes associated with good character are prioritized in the perceptual matching task. In a series of control experiments, we confirmed that moral content drove the prioritization effect, instead of other factors such as familiarity. In the subsequent experiments, we further tested whether the prioritization of moral character was caused by the valence of moral character or the interaction between valence and self-referential processing and found that only shapes associated with both good character and the self are prioritized, suggesting spontaneous moral self-referential as a novel mechanism underlying prioritization of good character in perceptual decision-making.

\hypertarget{disclosures}{%
\section{Disclosures}\label{disclosures}}

We reported all the measurements, analyses, and results in all the experiments in the current study. Participants whose overall accuracy lower than 60\% were excluded from analysis. Also, the accurate responses with less than 200ms reaction times were excluded from the analysis. These excluded data can be found in the shared raw data files.

All the experiments reported were not pre-registered. Most experiments (1a \textasciitilde{} 4b, except experiment 3b) reported in the current study were first finished between 2013 to 2016 in Tsinghua University, Beijing, China. Participants in these experiments were recruited in the local community. To increase the sample size of experiments to 50 or more (Simmons, Nelson, \& Simonsohn, 2013), we recruited additional participants in Wenzhou University, Wenzhou, China in 2017 for experiment 1a, 1b, 4a, and 4b. Experiment 3b was finished in Wenzhou University in 2017 (See Table S1 for overview of these experiments).

All participants received informed consent and compensated for their time. These experiments were approved by the ethic board in the Department of Psychology, Tsinghua University.

\hypertarget{general-methods}{%
\section{General methods}\label{general-methods}}

\hypertarget{design-and-procedure}{%
\subsection{Design and Procedure}\label{design-and-procedure}}

This series of experiments used the social associative learning paradigm (or self-tagging paradigm, see Sui, He, and Humphreys (2012)), in which participants first learned the associations between geometric shapes and labels of different moral characters (e.g., in the first three studies, the triangle, square, and circle and Chinese words for ``good person'', ``neutral person'', and ``bad person'', respectively). The associations of shapes and labels were counterbalanced across participants. The paradigm consists of a brief learning stage and a test stage. During the learning stage, participants were instructed about the association between shapes and labels. Participants started the test stage with a practice phase to familiarize themselves with the task, in which they viewed one of the shapes above the fixation while one of the labels below the fixation and judged whether the shape and the label matched the association they learned. If the overall accuracy reached 60\% or higher at the end of the practicing session, participants proceeded to the experimental task of the test stage. Otherwise, they finished another practices sessions until the overall accuracy was equal to or greater than 60\%. The experimental task shared the same trial structure as in the practice.

Experiments 1a, 1b, 1c, 2, 5, and 6a were designed to explore and confirm the effect of moral character on perceptual matching. All these experiments shared a 2 (matching: match vs.~nonmatch) by 3 (moral character: good vs.~neutral vs.~bad person) within-subject design. Experiment 1a was the first one of the whole series of studies, which aimed to examine the prioritization of moral character and found that shapes associated with good character were prioritized. Experiments 1b, 1c, and 2 were to confirm that it is the moral character that caused the effect. More specifically, experiment 1b used different Chinese words as labels to test whether the effect was contaminated by familiarity. Experiment 1c manipulated the moral character indirectly: participants first learned to associate different moral behaviors with different Chinese names, after remembering the association, they then associate the names with different shapes and finished the perceptual matching task. Experiment 2 further tested whether the way we presented the stimuli influence the prioritization of moral character, by sequentially presenting labels and shapes instead of simultaneous presentation. Note that a few participants in experiment 2 also participated in experiment 1a because we originally planned a cross-task comparison. Experiment 5 was designed to compare the prioritization of good character with other important social values (aesthetics and emotion). All social values had three levels, positive, neutral, and negative, and were associated with different shapes. Participants finished the associative learning task for different social values in different blocks, and the order of the social values was counterbalanced. Only the data from moral character blocks, which shared the design of experiment 1a, were reported here. Experiment 6a, which shared the same design as experiment 2, was an EEG experiment aimed at exploring the neural mechanism of the prioritization of good character. Only behavioral results of experiment 6a were reported here.

Experiments 3a, 3b, and 6b were designed to test whether the prioritization of good character can be explained by the valence effect alone or by an interaction between the valence effect and self-referential processing. To do so, we included self-reference as another within-subject variable. For example, experiment 3a extended experiment 1a into a 2 (matching: match vs.~nonmatch) by 2 (reference: self vs.~other) by 3 (moral character: good vs.~neutral vs.~bad) within-subject design. Thus, in experiment 3a, there were six conditions (good-self, neutral-self, bad-self, good-other, neutral-other, and bad-other) and six shapes (triangle, square, circle, diamond, pentagon, and trapezoids). Experiment 6b was an EEG experiment based on experiment 3a but presented the label and shape sequentially. Because of the relatively high working memory load (six label-shape pairs), participants finished experiment 6b in two days. On the first day, participants completed the perceptual matching task as a practice, and on the second day, they finished the task again while the EEG signals were recorded. We only focus on the first day's data here. Experiment 3b was designed to test whether the effect found in experiments 3a and 6b is robust if we separately present the self-referential trials and other-referential trials. That is, participants finished two different types of blocks: in the self-referential blocks, they only made matching judgments to shape-label pairs that related to the self (i.e., shapes and labels of good-self, neutral-self, and bad-self), in the other-referential blocks, they only responded to shape-label pairs that related to the other (i.e., shapes and labels of good-other, neutral-other, and bad-other).

Experiments 4a and 4b were designed to further test the interaction between valence and self-referential process in prioritization of good character. In experiment 4a, participants were instructed to learn the association between two shapes (circle and square) with two labels (self vs.~other) in the learning stage. In the test stage, they were instructed only respond to the shape and label during the test stage. To test the effect of moral character, we presented the labels of moral character in the shapes and instructed participants to ignore the words in shapes when making matching judgments. In the experiment 4b, we reversed the role of self and moral character in the task: Participants learned associations between three labels (good-person, neutral-person, and bad-person) and three shapes (circle, square, and triangle) and made matching judgments about the shape and label of moral character, while words related to identity, ``self'' or ``other'', were presented within the shapes. As in 4a, participants were told to ignore the words inside the shape during the perceptual matching task.

\hypertarget{stimuli-and-materials}{%
\subsection{Stimuli and Materials}\label{stimuli-and-materials}}

We used E-prime 2.0 for presenting stimuli and collecting behavioral responses. Data were collected from two universities located in two different cities in China. Participants recruited from Tsinghua University, Beijing, finished the experiment individually in a dim-lighted chamber. Stimuli were presented on 22-inch CRT monitors and participants rested their chins on a brace to fix the distance between their eyes and the screen around 60 cm. The visual angle of geometric shapes was about \(3.7^\circ × 3.7^\circ\), the fixation cross is of \(0.8^\circ × 0.8^\circ\) visual angle at the center of the screen. The words were of \(3.6^\circ\) × \(1.6^\circ\) visual angle. The distance between the center of shapes or images of labels and the fixation cross was of \(3.5^\circ\) visual angle. Participants from Wenzhou University, Wenzhou, finished the experiment in a group consisting of 3 \textasciitilde{} 12 participants in a dim-lighted testing room. They were instructed to finish the whole experiment independently. Also, they were told to start the experiment at the same time so that the distraction between participants was minimized. The stimuli were presented on 19-inch CRT monitors with the same set of parameters in E-prime 2.0 as in Tsinghua University, however, the visual angles could not be controlled because participants' chins were not fixed.

In most of these experiments, participants were also asked to fill out questionnaires after finishing the behavioral tasks. All the questionnaire data were open (see, dataset 4 in Liu et al., 2020). See Table 1 for a summary of information about all the experiments.

\begin{table}[tbp]

\begin{center}
\begin{threeparttable}

\caption{\label{tab:Table_1_exp_info}Information about all experiments.}

\begin{tabular}{llllllll}
\toprule
ExpID & \multicolumn{1}{c}{Time} & \multicolumn{1}{c}{Location} & \multicolumn{1}{c}{N} & \multicolumn{1}{c}{No..Trials.per.Cond.} & \multicolumn{1}{c}{Self.ref} & \multicolumn{1}{c}{Stim.for.Morality} & \multicolumn{1}{c}{Presenting.order}\\
\midrule
Exp\_1a\_1 & 2014-04 & Beijing & 38 (35) & 60 & NA & words & Simultaneously\\
Exp\_1a\_2 & 2017-04 & Wenzhou & 18 (16) & 60 & NA & words & Simultaneously\\
Exp\_1b\_1 & 2014-10 & Beijing & 39 (27) & NA & NA & words & Simultaneously\\
Exp\_1b\_2 & 2017-04 & Wenzhou & 33 (25) & NA & NA & words & Simultaneously\\
Exp\_1c & 2014-10 & Beijing & 23 (23) & NA & NA & descriptions & Simultaneously\\
Exp\_2 & 2014-05 & Beijing & 35 (34) & NA & NA & words & Sequentially\\
Exp\_3a & 2014-11 & Beijing & 38 (35) & NA & explicit & words & Simultaneously\\
Exp\_3b & 2017-04 & Wenzhou & 61 (56) & NA & explicit & words & Simultaneously\\
Exp\_4a\_1 & 2015-06 & Beijing & 32 (29) & NA & implicit & words & Simultaneously\\
Exp\_4a\_2 & 2017-04 & Wenzhou & 32 (30) & NA & implicit & words & Simultaneously\\
Exp\_4b\_1 & 2015-10 & Beijing & 34 (32) & NA & implicit & words & Simultaneously\\
Exp\_4b\_2 & 2017-04 & Wenzhou & 19 (13) & NA & implicit & words & Simultaneously\\
Exp\_5 & 2016-01 & Beijing & 43 (38) & NA & NA & words & Simultaneously\\
Exp\_6a & 2014-12 & Beijing & 24 (24) & NA & NA & words & Sequentially\\
Exp\_6b & 2016-01 & Beijing & 23 (22) & NA & explicit & words & Sequentially\\
Exp\_7a & 2016-07 & Beijing & 35 (29) & NA & explicit & words & Simultaneously\\
Exp\_7b & 2018-05 & Beijing & 46 (42) & NA & explicit & words & Simultaneously\\
\bottomrule
\addlinespace
\end{tabular}

\begin{tablenotes}[para]
\normalsize{\textit{Note.} Stim of Morality = How moral character was manipulated; Presenting order = how shapes \& labels were presented. The data from experiments 7a \& 7b, which were reported in Hu et al (2020), are only included in the meta-analysis in supplementary materials.}
\end{tablenotes}

\end{threeparttable}
\end{center}

\end{table}

\hypertarget{data-analysis}{%
\subsection{Data analysis}\label{data-analysis}}

We used the \texttt{tidyverse} of r (see script \texttt{Load\_save\_data.r}) to preprocess the data. The data from all experiments were then analyzed using Bayesian hierarchical models.

We used the Bayesian hierarchical model (BHM, or Bayesian generalized linear mixed models, Bayesian multilevel models) to model the reaction time and accuracy data because BHM provided three advantages over the classic NHST approach (repeated measure ANOVA or \emph{t}-tests). First, BHM estimates the posterior distributions of parameters for statistical inference, therefore providing uncertainty in estimation (Rouder \& Lu, 2005). Second, BHM, where generalized linear mixed models could be easily implemented, can use distributions that fit the distribution of real data instead of using the normal distribution for all data. Using appropriate distributions for the data will avoid misleading results and provide a better fitting of the data. For example, Reaction times are not normally distributed but are right skewed, and the linear assumption in ANOVAs is not satisfied (Rousselet \& Wilcox, 2019). Third, BHM provides a unified framework to analyze data from different levels and different sources, avoiding information loss when we need to combine data from different experiments.

We used the \texttt{r} package \texttt{BRMs} (Bürkner, 2017), which used Stan (Carpenter et al., 2017) as the back-end, for the BHM analyses. We estimated the overall effect across experiments that shared the same experimental design using one model, instead of a two-step approach that was adopted in mini-meta-analysis (e.g., Goh, Hall, \& Rosenthal, 2016). More specifically, a three-level model was used to estimate the overall effect of prioritization of good character, which included data from five experiments: 1a, 1b, 1c, 2, 5, and 6a. Similarly, a three-level HBM model is used for experiments 3a, 3b, and 6b. Results of individual experiments can be found in the supplementary results. For experiments 4a and 4b, which tested the implicit interaction between the self and good character, we used HBM for each experiment separately.

For questionnaire data, we only reported the subjective distance between different persons or moral character in the supplementary results and did not analyze other questionnaire data, which are described in (Liu et al., 2020).

\hypertarget{response-data}{%
\subsubsection{Response data}\label{response-data}}

We followed previous studies (Hu, Lan, Macrae, \& Sui, 2020; Sui et al., 2012) and used the signal detection theory approach to analyze the response data. More specifically, the match trials are treated as signals and non-match trials are noise. The sensitivity and criterion of signal detection theory are modeled through BHM (Rouder \& Lu, 2005).

We used the Bernoulli distribution for the signal detection theory. The probability that the \(j\)th subject responded ``match'' (\(y_{ij} = 1\)) at the \(i\)th trial \(p_{ij}\) is distributed as a Bernoulli distribution with parameter \(p_{ij}\):

\[ y_{ij} \sim Bernoulli(p_{ij})\]
The reparameterized value of \(p_{ij}\) is a linear regression of the independent variables:
\[ \Phi(p_{ij}) = 0 + \beta_{0j}Valence_{ij} + \beta_{1j}IsMatch_{ij} * Valence_{ij}\]
where the probits (z-scores; \(\Phi\), ``Phi'') of \(p\)s is used for the regression.

The subjective-specific intercepts (\(\beta_{0} = -zFAR\)) and slopes (\(\beta_{1} = d'\)) are described by multivariate normal with means and a covariance matrix for the parameters.
\[ \begin{bmatrix}\beta_{0j}\\
\beta_{1j}\\
\end{bmatrix} \sim N(\begin{bmatrix}\theta_{0}\\
\theta_{1}\\
\end{bmatrix}, \sum) \]

We used the following formula for experiments 1a, 1b, 1c, 2, 5, and 6a, which have a 2 (matching: match vs.~non-match) by 3 (moral character: good vs.~neutral vs.~bad) within-subject design:

\texttt{saymatch\ \textasciitilde{}\ 0\ +\ Valence\ +\ Valence:ismatch\ +\ (0\ +\ Valence\ +\ Valence:ismatch\ \textbar{}\ Subject)\ +\ (0\ +\ Valence\ +\ Valence:ismatch\ \textbar{}\ ExpID\_new:Subject)\ ,\ family\ =\ bernoulli(link="probit")}

in which the \texttt{saymatch} is the response data whether participants pressed the key corresponding to ``match'', \texttt{ismatch} is the independent variable of matching, \texttt{Valence} is the independent variable of moral character, \texttt{Subject} is the index of participants, and \texttt{Exp\_ID\_new} is the index of different experiments. Not that we distinguished data collected from two universities.

For experiments 3a, 3b, and 6b, an additional variable, i.e., reference (self vs.~other), was included in the formula:

\texttt{saymatch\ \textasciitilde{}\ 0\ +\ ID:Valence\ +\ ID:Valence:ismatch\ +\ (0\ +\ ID:Valence\ +\ ID:Valence:ismatch\ \textbar{}\ Subject)\ +\ (0\ +\ ID:Valence\ +\ ID:Valence:ismatch\ \textbar{}\ ExpID\_new:Subject),\ family\ =\ bernoulli(link="probit")}
in which the \texttt{ID} is the independent variable ``reference'', which means whether the stimulus was self-referential or other-referential.

\hypertarget{reaction-times}{%
\subsubsection{Reaction times}\label{reaction-times}}

We used log-normal distribution (\url{https://lindeloev.github.io/shiny-rt/\#34_(shifted)_log-normal}) to model the RT data. This means that we need to estimate the posterior of two parameters: \(\mu\), and \(\sigma\). \(\mu\) is the mean of the \texttt{logNormal} distribution, and \(\sigma\) is the disperse of the distribution.

The reaction time of the \(j\)th subject on \(i\)th trial, \(y_{ij}\), is log-normal distributed:
\[ log(y_{ij}) \sim N(\mu_{j}, \sigma_{j})\]

The parameter \(\mu_{j}\) is a linear regression of the independent variables:
\[\mu_{j} = \beta_{0j} + \beta_{1j}*IsMatch_{ij} * Valence_{ij}\]

and the parameter \(\sigma_{j}\) does not vary with independent variables:
\[\sigma_{j} \sim HalfNormal()\]

The subjective-specific intercepts (\(\beta_{0j}\)) and slopes (\(\beta_{1j}\)) are described by multivariate normal with means and a covariance matrix for the parameters.
\[ \begin{bmatrix}\beta_{0j}\\
\beta_{1j}\\
\end{bmatrix} \sim N(\begin{bmatrix}\theta_{0}\\
\theta_{1}\\
\end{bmatrix}, \sum) \]

The formula used for experiments 1a, 1b, 1c, 2, 5, and 6a, which have a 2 (matching: match vs.~non-match) by 3 (moral character: good vs.~neutral vs.~bad) within-subject design, is as follows:

\texttt{RT\_sec\ \textasciitilde{}\ 1\ +\ Valence*ismatch\ +\ (Valence*ismatch\ \textbar{}\ Subject)\ +\ (Valence*ismatch\ \textbar{}\ ExpID\_new:Subject),\ family\ =\ lognormal()}
in which \texttt{RT\_sec} is the reaction times data with the second as a unit. The other variables has the same meaning as the response data.

For experiments 3a, 3b, and 6b, which have a 2 by 2 by 3 within-subject design, the formula is as follows:
\texttt{RT\_sec\ \textasciitilde{}\ 1\ +\ ID*Valence\ +\ (ID*Valence\ \textbar{}\ Subject)\ +\ (ID*Valence\ \textbar{}\ ExpID\_new:Subject),\ family\ =\ lognormal()}

Note that for experiments 3a, 3b, and 6b, the three-level model for reaction times only included the matched trials to avoid divergence when estimating the posterior of the parameters.

\hypertarget{testing-hypotheses}{%
\subsubsection{Testing hypotheses}\label{testing-hypotheses}}

\hypertarget{prioritization-of-moral-character}{%
\paragraph{Prioritization of moral character}\label{prioritization-of-moral-character}}

We tested whether moral characters are prioritized by examining the population-level effects (also called fixed effect) of the three-level Bayesian hierarchical model of experiments 1a, 1b, 1c, 2, 5, and 6a. More specifically, we calculated the difference between the posterior distribution of the good/bad character and the neutral character and tested whether the 95\% highest density intervals (HDIs) of the difference include zero. If the 95\% highest density intervals do not include zero, we infer that there is a population-level difference between the conditions in the test, otherwise, we will infer that there is no evidence for such a difference. Note that for reaction times, we focused on the matched trials as in previous studies.

\hypertarget{modulation-of-self-referential-processing}{%
\paragraph{Modulation of self-referential processing}\label{modulation-of-self-referential-processing}}

We tested the modulation effect ofself-referential processing by examining the interaction between moral character and self-referential process for the three-level Bayesian hierarchical model of experiments 3a, 3b, and 6b. More specifically, we tested two possible explanations for the prioritization of good character: pure valence effect or the valence effect is modulated by self-referential process. If the former is correct, then there will be no interaction between moral character and self-referential processing, i.e., the prioritization effect exhibits a similar pattern for both self- and other-referential conditions. On the other hand, if spontaneous self-referential processing account is true, then there will be an interaction between the two factors, i.e., the prioritization effect exhibits different patterns for self- and other-referential conditions.

\hypertarget{spontaneous-binding-between-the-self-and-good-character}{%
\paragraph{Spontaneous binding between the self and good character}\label{spontaneous-binding-between-the-self-and-good-character}}

For on data from experiments 4a and 4b, we further examined whether the self-referential processing for moral characters is spontaneous (i.e., whehter the good character is spontaneously bound with self). For experiment 4a, if there exists a spontaneous binding between self and good character, there should be an interaction between moral character and self-referential processing, e.g., the task-irrelevant moral words either facilitate or slows down the response to self- or other-referential conditions. For experiment 4b, if there exists a spontaneous binding between self and good character, then, there will be a self-other difference for some moral character conditions but not for other moral character conditions.

\hypertarget{results}{%
\section{Results}\label{results}}

\hypertarget{prioritization-of-good-character}{%
\subsection{Prioritization of good character}\label{prioritization-of-good-character}}

In this part, we report results from experiments 1a, 1b, 1c, 2, 5, and 6a. All these experiments shared similar design and can be used for testing the prioritization effect of moral character. The valid and unique sample size is 192. Note that for both experiment 1a and 1b, there were two datasets that were collected at different time points and location, thus we treated them as independent samples. Three-level Bayesian models revealed a robust prioritization effect of good character in perceptual matching task. The detailed results of each experiment can be found in supplementary materials.

\begin{figure}
\centering
\includegraphics{Notebook_Pos_Self_Salience_APA_files/figure-latex/plot-bayes-meta-1-1.pdf}
\caption{\label{fig:plot-bayes-meta-1}Effect of moral character on RT and d'}
\end{figure}

For the \emph{d} prime, we found robust effect of moral character. Shapes associated with good character (``good person'', ``kind person'' or a name associated with morally good behavioral history) has higher sensitivity (median = 2.51, 95\% HDI = {[}2.23 2.78{]}) than shapes associated with neutral character (median = 2.19, 95\% HDI = {[}1.88 2.50{]}), \(median_{diff}\) = 0.31, 95\% HDI {[}0.00 0.64{]} , but we did not find differences between shapes associated with bad character (median = 2.25, 95\% HDI = {[}1.94 2.55{]}) and neutral character, \(median_{diff}\) = 0.05, 95\% HDI {[}-0.28 0.39{]}.

For the reaction times, we also found robust effect of moral character for both match trials (see figure \ref{fig:plot-bayes-meta-1} C) and nonmatch trials (\textbf{see supplementary materials}). For match trials, shapes associated with good character has faster responses (median = 579.03 ms, 95\% HDI = {[}500.20 660.89{]}) than shapes associated with neutral character (median = 623.59 ms, 95\% HDI = {[}542.83 710.82{]}), \(median_{diff}\) = -44.19, 95\% HDI {[}-59.85 -30.36{]}. We also found that the responses to shapes associated with bad character (median = 640.86 ms, 95\% HDI = {[}561.22 729.99{]}) were slower as compared to the neutral character, \(median_{diff}\) = 16.85, 95\% HDI {[}2.82 30.10{]}. See Figure \ref{fig:plot-bayes-meta-1}.

For the nonmatch trials, we found the advantage of good character: Shapes associated with good character (median = 654.16 ms, 95\% HDI = {[}573.12 742.91{]}) are faster than shapes associated with neutral (median = 671.81 ms, 95\% HDI = {[}588.33 762.65{]}), \(median_{diff}\) = -17.72 ms, 95\% HDI {[}-24.58 -11.19{]}. Similarly, the shapes associated with bad character (median = 676.93 ms, 95\% HDI = {[}590.23 765.67{]}) was responded slower than shapes associated with neutral character, \(median_{diff}\) = 16.85 ms, 95\% HDI {[}2.82 30.10{]}, but the effect size was smaller than the match trials (\textbf{see supplementary materials}).

\hypertarget{self-referential-process-modulates-prioritization-of-good-character}{%
\subsection{Self-referential process modulates prioritization of good character}\label{self-referential-process-modulates-prioritization-of-good-character}}

Here we report the results from three-level Bayesian models of three experiments (3a, 3b, and 6b) that aimed at testing whether the prioritization effect of good character is modulated by self-referential processing. These three experiments included data from 108 unique participants.

To test the modulation effect, we focused on the differences between good/bad character with neutral character for self-referential and other-referential separately.Also, we compared the differences between the difference, i.e., how did the differences between good and bad characters under the self-referential conditions differ from that under other-referential conditions. The detailed results of each experiment can be found in supplementary materials.

\begin{figure}
\centering
\includegraphics{Notebook_Pos_Self_Salience_APA_files/figure-latex/plot-bayes2-1.pdf}
\caption{\label{fig:plot-bayes2}Interaction between moral character and self-referential}
\end{figure}

For the \emph{d} prime, we found that an interaction between moral character effect and self-referential processing: the good-neutral differences is larger for self-referential condition than for the other-referential condition (\(median_{diff}\) = 0.52; 95\% HDI = {[}-1.54 2.72{]}). However, this is not the case for the bad-neutral differences (\(median_{diff}\) = -0.01; 95\% HDI = {[}-1.55 2.37{]}). Further analyses revealed that prioritization effect of good character (as compared to neutral) only appeared for self-referential conditions but not other-referential conditions. The estimated \emph{d} prime for good-self was greater than neutral-self (\(median_{diff}\) = 0.57; 95\% HDI = {[}-0.93 2.88{]}), \emph{d} prime for good-self was also greater than good-other condition (\(median_{diff}\) = ; 95\% HDI = {[} {]}). The differences between bad-self and neutral-self, good-other and neutral-other, bad-other and neutral-other are all centered around zero (see Figure \ref{fig:plot-bayes2}, B, D).

For the RTs of matched trials, we found an interaction between moral character and self-referential processing: the good-neutral differences were different for the self- and other-referential conditions (\(median_{diff}\) = -155.11; 95\% HDI = {[}-755.36 395.38{]}). However, this was not the case for bad-neutral differences (\(median_{diff}\) = -55.63; 95\% HDI = {[}-604.70 561.00{]}). Further analyses revealed a robust good-self prioritization effect as compared to neutral-self (\(median_{diff}\) = -57.91; 95\% HDI = {[}-224.43 41.14{]}) and good-other (\(median_{diff}\) = -107.21; 95\% HDI = {[}-369.05 92.95{]}) conditions. Similar to the results of \emph{d'}, we found that both good character and bad character were responded slower than neutral character for other-referential conditions. See Figure \ref{fig:plot-bayes2}.

These results suggested that the prioritization of good character is not solely driven by valence of moral character. Instead, the self-referential processing modulated the prioritization of good character: the good character was prioritized only when it was self-referential. When the moral character was other-referential, good character has similar performance as bad character.

\hypertarget{spontaneous-binding-between-the-good-character-and-the-self}{%
\subsection{Spontaneous binding between the good character and the self}\label{spontaneous-binding-between-the-good-character-and-the-self}}

Here we reported the results from experiments 4a and 4b that aimed at testing whether the binding between self and good character happen even when these two piece of information are separated and only one of them is task-relevant. We are interested in testing whether the task-relevance modulated the effect observed in previous experiment.

In experiment 4a, where self- and other-referential were task-relevant and moral character are task-irrelevant, we found self-related conditions were performed better than other-related conditions, on both \emph{d} prime and reaction times. This pattern is consistent with previous studies (e.g., Sui et al. (2012)).

More importantly, we found evidence, albeit weak, that task-irrelevant moral character also played an role. For shapes associated with self, \emph{d'} was greater when shapes had a good character inside the shape (median = 2.83, 95\% HDI {[}2.63 3.01{]}) than shapes that have neutral character (median = 2.74, 95\% HDI {[}2.58 2.95{]}, BF = 4.4) or bad character (median = 2.76, 95\% HDI {[}2.56 2.95{]}, BF = 3.1), but we did not found difference between shapes with bad character and neutral character inside for the self-referential shapes. For shapes associated with other, the results of \emph{d'} revealed a reversed pattern to the self-referential condition: \emph{d} prime was smaller when shapes had a good character inside (median = 1.87, 95\% HDI {[}1.71 2.04{]}) than had neutral (median = 1.96, 95\% HDI {[}1.80 2.14{]}) or bad character (median = 1.98, 95\% HDI {[}1.79 2.17{]}) inside. See Figure \ref{fig:plot-exp4-all}.

A similar pattern was found for RTs in matched trials. For self-referential condition, when good character was presented as a task-irrelevant stimuli, the responds (median = 641, 95\% HDI {[}623 662{]}) were faster than when neutral character (median = 649, 95\% HDI {[}631 668{]}) or bad character (median = 648, 95\% HDI {[}628 667{]}) were inside. This effect was reversed for other-referential condition: shapes associated with other with good character inside (median = 733, 95\% HDI {[}711 754{]}) were slower than with neutral character (median = 721, 95\% HDI {[}702 741{]}) or bad character (median = 718, 95\% HDI {[}696 740{]}) inside.

\begin{figure}
\centering
\includegraphics{Notebook_Pos_Self_Salience_APA_files/figure-latex/plot-exp4-all-1.pdf}
\caption{\label{fig:plot-exp4-all}Experiment 4: Implicit binding between good character and the self.}
\end{figure}

In experiment 4b, moral character was the task-relevant factor, and we found that there were main effect of moral character: shapes associated with good character were performed better than other-related conditions, on both \emph{d'} and reaction times. This pattern shows a robust prioritization effect of good character.

Most importantly, we found evidence that task-irrelevant self-referential process also played an role. For shapes associated with good person, the \emph{d} prime was greater when shapes had an ``self'' inside than with ``other'' inside (\(mean_{diff}\) = 0.14, 95\% credible intervals {[}-0.02, 0.31{]}, BF = 12.07), but this effect did not happen when the target shape where associated with ``neutral'' (\(mean_{diff}\) = 0.04, 95\% HDI {[}-.11, .18{]}) or ``bad'' person (\(mean_{diff}\) = -.05, 95\% HDI{[}-.18, .09{]}).

The same trend appeared for the RT data. For shapes associated with good person, with a ``self'' inside the shape reduced the reaction times as compared with when a ``other'' inside the shape (\(mean_{diff}\) = -55 ms, 95\% HDI {[}-75, -35{]}), but this effect did not occur when the shapes were associated neutral (\(mean_{diff}\) = 10, 95\% HDI {[}1, 20{]}) or bad (\(mean_{diff}\) = 5, 95\% HDI {[}-16, 27{]}) person. See Figure \ref{fig:plot-exp4-all}.

\hypertarget{discussion}{%
\section{Discussion}\label{discussion}}

Across nine experiments, we explored the prioritization effect of moral character and the underlying mechanism by a combination of social associative learning and perceptual matching task. First, we found robust effect that good character was prioritized in the shape-label matching task across five experiments. Second, across three experiments, we found that the prioritization of good character was not solely driven by moral valence itself, i.e., ``good'' vs ``bad''. Instead, this effect was modulated by a self-referential processing: prioritization only occurred when moral characters are self-referential. Finally, the prioritization of the combination of good character and self occured, albeit weak, even when either the self- or character- related information was irrelevant to experimental task (experiment 4a and 4b). In contrast, for other referential condition, explicitly or implicitly, the performance of good character was worse than neutral character. Together, these results highlighted the importance of the self in perceiving more character information, suggested a spontaneous self-referential process when moral character is involved in perceptual decision-making. These results contribute to a growing literature on the social and relational nature of perception {[}Xiao, Coppin, and Bavel (2016); Freeman, Stolier, and Brooks (2020); @hafri\_perception\_2021{]}.

The evidence for the effect of moral character on perceptual decision-making is robust in our study. Previous studies reported the effect of morality on perception but the results and the mechanisms were disputed. For example, (Anderson et al., 2011) reported that faces associated with bad social behavior capture attention more rapidly, however, an independent team failed to replicate the effect (Stein et al., 2017). Another studies by Gantman and Van Bavel (2014) found that moral words are more likely to be judged as words when it was presented subliminally, however, this effect may caused by semantic priming instead of morality (Firestone \& Scholl, 2015; Jussim et al., 2016). In the current study, the associative learning task allowed us to eliminate the semantic priming. First, only a few pairs of stimuli was used and different stimuli were different in many dimensions, makes it impossible for priming between them. Second, stimuli that used to represent moral character are neutral stimuli before the associative learning. Moreover, in experiment 1c where participants first associate moral behaviors with neutral names and then paired names with neutral shapes, we still found that effect of moral character, suggesting that it is the moral content instead of other features of labels/names that drives the prioritization effect. Finally, all eight different samples in the first set of experiments are consistent, this further confirmed that the prioritization effect of good character found in our paradigm is robust.

The prioritization of good character found in the current study was incongruent with previous moral perception studies, which usually reported a negativity effect, i.e., information related to bad moral character are processed better (Anderson et al., 2011; Eiserbeck \& Abdel Rahman, 2020). This discrepancy may resulted from differences in the experimental task: while in many previous moral perception studies, the participants were asked to detect the existence of a stimuli, the current task asked participants to recognize a pattern. In other words, previous studies targeted early stages of perception while the current task focus more on the decision-making at relative later stage of information processing. This discrepancy is consistent with the pattern found in studies with emotional stimuli (Pool, Brosch, Delplanque, \& Sander, 2016).

We expanded previous moral perception studies by examining two possible explanations of the prioritization of good character: valence effect only or an interaction between valence and self-referential processing. Our results revealed that prioritization of good character is modulated by self-referential process: the good character was prioritized when it was related to self, even when the self-relatedness was task irrelevant. By contrast, good character information was no longer prioritized when it was associated with other. The modulation effect of self-referential processing was large when the relationship between moral character and the self was explicit. More importantly, the effect persisted when the relationship between moral character and the self information was implicit, suggesting a spontaneous self-referential processing when both information were presented. An possible explanation for this spontaneous self-referential of good character is that moral-self is central to our identity (Freitas, Cikara, Grossmann, \& Schlegel, 2017; Strohminger, Knobe, \& Newman, 2017) and the motivation to maintain a moral-self view also influenced the perceptual decision-making.

Although the results here revealed prioritization of good character in perceptual decision-making, we did not claim that the moral-self motivation \emph{penetrates} perception. Perceptual decision-making process include processes more than just encoding the sensory inputs but also other processes. To fully account the nuance of behavioral data and/or related data collected from other modules {[}{]}, we need computational models and integrative experimental approach (Almaatouq et al., 2022). For example, sequential sampling models suggest that, when making a perceptual decision, the agent is continuously accumulate evidence until the amount of evidence passed a threshold, then a decision is made (Chuan-Peng et al., 2022; Forstmann, Ratcliff, \& Wagenmakers, 2016; Ratcliff, Smith, Brown, \& McKoon, 2016). In these models, the evidence, or decision variable, can accumulate from both sensory information but also memory (Shadlen \& Shohamy, 2016). Recently, applications of sequential sample model to perceptual matching tasks also suggest that different processes may contributed to the prioritization effect of self (Golubickis et al., 2017) or good self (Hu et al., 2020). Similarly, reinforcement learning models also revealed that the key difference between self- and other-referential learning lies in the learning rate (Lockwood et al., 2018). These studies suggest that computational models are need to disentangle the cognitive processes underlying the prioritization of good character.

\hypertarget{references}{%
\section{References}\label{references}}

\begingroup
\setlength{\parindent}{-0.5in}
\setlength{\leftskip}{0.5in}

\hypertarget{refs}{}
\begin{CSLReferences}{1}{0}
\leavevmode\vadjust pre{\hypertarget{ref-Almaatouq_BBS_2022}{}}%
Almaatouq, A., Griffiths, T. L., Suchow, J. W., Whiting, M. E., Evans, J., \& Watts, D. J. (2022). \emph{Beyond {Playing} 20 {Questions} with {Nature}: {Integrative} {Experiment} {Design} in the {Social} and {Behavioral} {Sciences}}. 1--55. \url{https://doi.org/10.1017/S0140525X22002874}

\leavevmode\vadjust pre{\hypertarget{ref-anderson_visual_2011}{}}%
Anderson, E., Siegel, E. H., Bliss-Moreau, E., \& Barrett, L. F. (2011). The visual impact of gossip. \emph{Science}, \emph{332}(6036), 1446--1448. \url{https://doi.org/10.1126/science.1201574}

\leavevmode\vadjust pre{\hypertarget{ref-Buxfcrkner_2017}{}}%
Bürkner, P.-C. (2017). Brms: An r package for bayesian multilevel models using stan {[}Journal Article{]}. \emph{Journal of Statistical Software; Vol 1, Issue 1 (2017)}. Retrieved from \href{https://www.jstatsoft.org/v080/i01\%0Ahttp://dx.doi.org/10.18637/jss.v080.i01}{https://www.jstatsoft.org/v080/i01
http://dx.doi.org/10.18637/jss.v080.i01}

\leavevmode\vadjust pre{\hypertarget{ref-Carpenter_2017_stan}{}}%
Carpenter, B., Gelman, A., Hoffman, M. D., Lee, D., Goodrich, B., Betancourt, M., \ldots{} Riddell, A. (2017). Stan: A probabilistic programming language {[}Journal Article{]}. \emph{Journal of Statistical Software}, \emph{76}(1). \url{https://doi.org/10.18637/jss.v076.i01}

\leavevmode\vadjust pre{\hypertarget{ref-Hu_hitchhikers_2022}{}}%
Chuan-Peng, H., Geng, H., Zhang, L., Fengler, A., Frank, M., \& Zhang, R.-Y. (2022). \emph{A {Hitchhiker}'s {Guide} to {Bayesian} {Hierarchical} {Drift}-{Diffusion} {Modeling} with {dockerHDDM}}. PsyArXiv. \url{https://doi.org/10.31234/osf.io/6uzga}

\leavevmode\vadjust pre{\hypertarget{ref-dunbar_gossip_2004}{}}%
Dunbar, R. I. M. (2004). Gossip in evolutionary perspective. \emph{Review of General Psychology}, \emph{8}(2), 100--110. \url{https://doi.org/10.1037/1089-2680.8.2.100}

\leavevmode\vadjust pre{\hypertarget{ref-eiserbeck_visual_2020}{}}%
Eiserbeck, A., \& Abdel Rahman, R. (2020). Visual consciousness of faces in the attentional blink: Knowledge-based effects of trustworthiness dominate over appearance-based impressions. \emph{Consciousness and Cognition}, \emph{83}, 102977. \url{https://doi.org/10.1016/j.concog.2020.102977}

\leavevmode\vadjust pre{\hypertarget{ref-ellemers_morality_2018}{}}%
Ellemers, N. (2018). Morality and social identity. In M. van Zomeren \& J. F. Dovidio (Eds.), \emph{The oxford handbook of the human essence} (pp. 147--158). New York, {NY}, {US}: Oxford University Press.

\leavevmode\vadjust pre{\hypertarget{ref-firestone_cognition_2015}{}}%
Firestone, C., \& Scholl, B. J. (2015). Enhanced visual awareness for morality and pajamas? Perception vs. Memory in "top-down" effects. \emph{Cognition}, \emph{136}, 409--416. \url{https://doi.org/10.1016/j.cognition.2014.10.014}

\leavevmode\vadjust pre{\hypertarget{ref-Firestone_2016_BBS}{}}%
Firestone, C., \& Scholl, B. J. (2016). Cognition does not affect perception: {Evaluating} the evidence for {``top-down''} effects. \emph{Behavioral and Brain Sciences}, \emph{39}, e229. \url{https://doi.org/10.1017/S0140525X15000965}

\leavevmode\vadjust pre{\hypertarget{ref-forstmann_ann_rev_2016}{}}%
Forstmann, B. U., Ratcliff, R., \& Wagenmakers, E.-J. (2016). Sequential {Sampling} {Models} in {Cognitive} {Neuroscience}: {Advantages}, {Applications}, and {Extensions}. \emph{Annual Review of Psychology}, \emph{67}(1). \url{https://doi.org/10.1146/annurev-psych-122414-033645}

\leavevmode\vadjust pre{\hypertarget{ref-freeman_chapter_2020}{}}%
Freeman, J. B., Stolier, R. M., \& Brooks, J. A. (2020). Chapter five - dynamic interactive theory as a domain-general account of social perception. In B. Gawronski (Ed.), \emph{Advances in experimental social psychology} (Vol. 61, pp. 237--287). Academic Press. \url{https://doi.org/10.1016/bs.aesp.2019.09.005}

\leavevmode\vadjust pre{\hypertarget{ref-freitas_origins_2017}{}}%
Freitas, J. D., Cikara, M., Grossmann, I., \& Schlegel, R. (2017). Origins of the belief in good true selves. \emph{Trends in Cognitive Sciences}, \emph{21}(9), 634--636. \url{https://doi.org/10.1016/j.tics.2017.05.009}

\leavevmode\vadjust pre{\hypertarget{ref-gantman_moral_2015}{}}%
Gantman, A. P., \& Bavel, J. J. V. (2015). Moral {Perception}. \emph{Trends in Cognitive Sciences}, \emph{19}(11), 631--633. \url{https://doi.org/10.1016/j.tics.2015.08.004}

\leavevmode\vadjust pre{\hypertarget{ref-gantman_see_2016}{}}%
Gantman, A. P., \& Bavel, J. J. V. (2016). See for {Yourself}: {Perception} {Is} {Attuned} to {Morality}. \emph{Trends in Cognitive Sciences}, \emph{20}(2), 76--77. \url{https://doi.org/10.1016/j.tics.2015.12.001}

\leavevmode\vadjust pre{\hypertarget{ref-gantman_moral_2014}{}}%
Gantman, A. P., \& Van Bavel, J. J. (2014). The moral pop-out effect: Enhanced perceptual awareness of morally relevant stimuli. \emph{Cognition}, \emph{132}(1), 22--29. \url{https://doi.org/10.1016/j.cognition.2014.02.007}

\leavevmode\vadjust pre{\hypertarget{ref-Goh_2016_mini}{}}%
Goh, J. X., Hall, J. A., \& Rosenthal, R. (2016). Mini meta-analysis of your own studies: Some arguments on why and a primer on how {[}Journal Article{]}. \emph{Social and Personality Psychology Compass}, \emph{10}(10), 535--549. \url{https://doi.org/10.1111/spc3.12267}

\leavevmode\vadjust pre{\hypertarget{ref-golubickis_self-prioritization_2017}{}}%
Golubickis, M., Falben, J. K., Sahraie, A., Visokomogilski, A., Cunningham, W. A., Sui, J., \& Macrae, C. N. (2017). Self-prioritization and perceptual matching: The effects of temporal construal. \emph{Memory \& Cognition}, \emph{45}(7), 1223--1239. \url{https://doi.org/10.3758/s13421-017-0722-3}

\leavevmode\vadjust pre{\hypertarget{ref-goodwin_moral_2015}{}}%
Goodwin, G. P. (2015). Moral character in person perception. \emph{Current Directions in Psychological Science}, \emph{24}(1), 38--44. \url{https://doi.org/10.1177/0963721414550709}

\leavevmode\vadjust pre{\hypertarget{ref-goodwin_moral_2014}{}}%
Goodwin, G. P., Piazza, J., \& Rozin, P. (2014). Moral character predominates in person perception and evaluation. \emph{Journal of Personality and Social Psychology}, \emph{106}(1), 148--168. \url{https://doi.org/10.1037/a0034726}

\leavevmode\vadjust pre{\hypertarget{ref-Hu_2020_GoodSelf}{}}%
Hu, C.-P., Lan, Y., Macrae, C. N., \& Sui, J. (2020). Good me bad me: Does valence influence self-prioritization during perceptual decision-making? {[}Journal Article{]}. \emph{Collabra: Psychology}, \emph{6}(1), 20. \url{https://doi.org/10.1525/collabra.301}

\leavevmode\vadjust pre{\hypertarget{ref-juechems_where_2019}{}}%
Juechems, K., \& Summerfield, C. (2019). Where does value come from? \emph{Trends in Cognitive Sciences}, \emph{23}(10), 836--850. \url{https://doi.org/10.1016/j.tics.2019.07.012}

\leavevmode\vadjust pre{\hypertarget{ref-jussim_interpretations_2016}{}}%
Jussim, L., Crawford, J. T., Anglin, S. M., Stevens, S. T., \& Duarte, J. L. (2016). Interpretations and methods: {Towards} a more effectively self-correcting social psychology. \emph{Journal of Experimental Social Psychology}, \emph{66}, 116--133. \url{https://doi.org/10.1016/j.jesp.2015.10.003}

\leavevmode\vadjust pre{\hypertarget{ref-Liu_2020_JOPD}{}}%
Liu, Q., Wang, F., Yan, W., Peng, K., Sui, J., \& Hu, C.-P. (2020). Questionnaire data from the revision of a chinese version of free will and determinism plus scale {[}Journal Article{]}. \emph{Journal of Open Psychology Data}, \emph{8}(1), 1. \url{https://doi.org/10.5334/jopd.49/}

\leavevmode\vadjust pre{\hypertarget{ref-lockwood_nat_comm_2018}{}}%
Lockwood, P. L., Wittmann, M. K., Apps, M. A. J., Klein-FlÃŒgge, M. C., Crockett, M. J., Humphreys, G. W., \& Rushworth, M. F. S. (2018). Neural mechanisms for learning self and other ownership. \url{https://doi.org/10.1038/s41467-018-07231-9}

\leavevmode\vadjust pre{\hypertarget{ref-pool_attentional_2016}{}}%
Pool, E., Brosch, T., Delplanque, S., \& Sander, D. (2016). Attentional bias for positive emotional stimuli: A meta-analytic investigation. \url{https://doi.org/10.1037/bul0000026}

\leavevmode\vadjust pre{\hypertarget{ref-ratcliff_tics_2016}{}}%
Ratcliff, R., Smith, P. L., Brown, S. D., \& McKoon, G. (2016). Diffusion {Decision} {Model}: {Current} {Issues} and {History}. \emph{Trends in Cognitive Sciences}, \emph{20}(4), 260--281. \url{https://doi.org/10.1016/j.tics.2016.01.007}

\leavevmode\vadjust pre{\hypertarget{ref-Rouder_2005_BHM_SDT}{}}%
Rouder, J. N., \& Lu, J. (2005). An introduction to bayesian hierarchical models with an application in the theory of signal detection {[}Journal Article{]}. \emph{Psychonomic Bulletin \& Review}, \emph{12}(4), 573--604. \url{https://doi.org/10.3758/bf03196750}

\leavevmode\vadjust pre{\hypertarget{ref-Rousselet_2019}{}}%
Rousselet, G. A., \& Wilcox, R. R. (2019). Reaction times and other skewed distributions: Problems with the mean and the median {[}Preprint{]}. \emph{Meta-Psychology}. \url{https://doi.org/10.1101/383935}

\leavevmode\vadjust pre{\hypertarget{ref-shadlen_decision_2016}{}}%
Shadlen, M. N., \& Shohamy, D. (2016). Decision {Making} and {Sequential} {Sampling} from {Memory}. \emph{Neuron}, \emph{90}(5), 927--939. \url{https://doi.org/10.1016/j.neuron.2016.04.036}

\leavevmode\vadjust pre{\hypertarget{ref-shore_social_2013}{}}%
Shore, D. M., \& Heerey, E. A. (2013). Do social utility judgments influence attentional processing? \emph{Cognition}, \emph{129}(1), 114--122. \url{https://doi.org/10.1016/j.cognition.2013.06.011}

\leavevmode\vadjust pre{\hypertarget{ref-Simmons_2013_life}{}}%
Simmons, J. P., Nelson, L. D., \& Simonsohn, U. (2013). \emph{Life after p-hacking} {[}Conference Proceedings{]}. \url{https://doi.org/10.2139/ssrn.2205186}

\leavevmode\vadjust pre{\hypertarget{ref-stein_no_2017}{}}%
Stein, T., Grubb, C., Bertrand, M., Suh, S. M., \& Verosky, S. C. (2017). No impact of affective person knowledge on visual awareness: Evidence from binocular rivalry and continuous flash suppression. \emph{Emotion}, \emph{17}(8), 1199--1207. \url{https://doi.org/10.1037/emo0000305}

\leavevmode\vadjust pre{\hypertarget{ref-strohminger_true_2017}{}}%
Strohminger, N., Knobe, J., \& Newman, G. (2017). The true self: A psychological concept distinct from the self: \emph{Perspectives on Psychological Science}. \url{https://doi.org/10.1177/1745691616689495}

\leavevmode\vadjust pre{\hypertarget{ref-Sui_2012_JEPHPP}{}}%
Sui, J., He, X., \& Humphreys, G. W. (2012). Perceptual effects of social salience: Evidence from self-prioritization effects on perceptual matching {[}Journal Article{]}. \emph{Journal of Experimental Psychology: Human Perception and Performance}, \emph{38}(5), 1105--1117. \url{https://doi.org/10.1037/a0029792}

\leavevmode\vadjust pre{\hypertarget{ref-sui_tics_2015}{}}%
Sui, J., \& Humphreys, G. W. (2015). The {Integrative} {Self}: {How} {Self}-{Reference} {Integrates} {Perception} and {Memory}. \emph{Trends in Cognitive Sciences}, \emph{19}(12), 719--728. \url{https://doi.org/10.1016/j.tics.2015.08.015}

\leavevmode\vadjust pre{\hypertarget{ref-sui_self_attention_2019}{}}%
Sui, J., \& Rotshtein, P. (2019). Self-prioritization and the attentional systems. \emph{Current Opinion in Psychology}, \emph{29}, 148--152. \url{https://doi.org/10.1016/j.copsyc.2019.02.010}

\leavevmode\vadjust pre{\hypertarget{ref-unkelbach_chapter_2020}{}}%
Unkelbach, C., Alves, H., \& Koch, A. (2020). Chapter three - negativity bias, positivity bias, and valence asymmetries: Explaining the differential processing of positive and negative information. In B. Gawronski (Ed.), \emph{Advances in experimental social psychology} (Vol. 62, pp. 115--187). Academic Press. \url{https://doi.org/10.1016/bs.aesp.2020.04.005}

\leavevmode\vadjust pre{\hypertarget{ref-xiao_perceiving_2016}{}}%
Xiao, Y. J., Coppin, G., \& Bavel, J. J. V. (2016). Perceiving the world through group-colored glasses: A perceptual model of intergroup relations. \emph{Psychological Inquiry}, \emph{27}(4), 255--274. \url{https://doi.org/10.1080/1047840X.2016.1199221}

\end{CSLReferences}

\endgroup


\end{document}
